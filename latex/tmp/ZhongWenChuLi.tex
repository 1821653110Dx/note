\documentclass[12pt ,a4paper, oneside]{ctexart}

\newcommand\degree{^\circ}	% 定义新命令\degree, \degree = ^\circ 

\usepackage{geometry}
\geometry{	% 设置页边距
left=2.5cm,	% 左边距为2.5cm
right=2.5cm,
top=2.5cm,
bottom=2.5cm,
}

\title{\heiti 勾股定理}	% 设置标题为‘勾股定理’, 字号为‘黑体’
\author{\kaishu 富理思•本}
\date{2023年9月4日}

\begin{document}

\maketitle

勾股定理可以用现代语言描述:

直角三角形斜边的平方等于靓腰的平方和。

可以用符号语言描述为:设直角三角形$ABC$,其中$$\angle C=90\degree$$,则有:
\begin{equation}	% 定义自动编号的行间公式
AB^2 = BC^2 + AC^2
\end{equation}


% here is my big formula
Let $f(x)$ be defined by the formula $$f(x) = 3x^2 + x - 1$$ which is a polynomial of degree 2.

\end{document}
