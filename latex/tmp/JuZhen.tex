\documentclass{ctexart}

\usepackage{amsmath}


\begin{document}
	\[			% 开始创建矩阵
	\begin{matrix}		% 创建没有括号的矩阵
		0 & 1 \\
		1 & 0	
	\end{matrix}		% 结束创建
	\qquad			% 插入空格
	\begin{pmatrix}		%
		0 & 1 \\
		1 & 0	
	\end{pmatrix}		% 创建带有小括号的矩阵
	\qquad
	\begin{bmatrix}		% 创建带有中括号的矩阵	
		0 & -1 \\
		1 & 0
	\end{bmatrix}
	\qquad
	\begin{Bmatrix}		% 创建带有大括号的矩阵	
		0 & -1 \\
		1 & 0
	\end{Bmatrix}
	\qquad
	\begin{vmatrix}		% 创建带有单竖线的矩阵
		0 & -1 \\
		1 & 0
	\end{vmatrix}
	\qquad
	\begin{Vmatrix}		% 创建带有双竖线的矩阵
		0 & -1 \\
		1 & 0
	\end{Vmatrix}
	\]		% 结束创建矩阵

	\[
	A = \begin{bmatrix}	% A=代括号的矩阵 
		a_{11} & \dots & a_{1n} \\
		 & \ddots & \vdots \\
		0 & & a_{nn}
	\end{bmatrix}_{n \times n}	% 矩阵带下标
	\]			
\end{document}
