\documentclass{article}		% 文章类:论文

\usepackage{ctex}	% 支持中文
\usepackage{amsmath}	

\begin{document}	% 开始正文
	\section{简介}	% 大标题:简介
		\LaTeX{}将排版内容分为文本模式和数学模式,文本模式用于普通文本排版,数学模式用于数学公式排版。
	\section{行间公式}
		\subsection{美元符号}	% 二级标题:美元符号
			交换律是 $a+b=b+a$,如 $1+2=2+1=3$
		\subsection{小括号}
			交换律是 \(a+b=b+a\),如 \(1+2=2+1=3\)
		\subsection{math环境}
			交换律是 \begin{math}a+b=b+a\end{math},如 \begin{math}1+2=2+1=3\end{math}
		\section{上下标}
			\subsection{上标}
				$3x^{20} - x + 2 = 0$
			\subsection{下标}
				$a_0, a_1, a_2, ... , a_{100}$
		\section{希腊字母}
			$\alpha$
			$\beta$
			$\gamma$
			$\epsilon$
			$\omega$

			$\Gamma$
			$\Delta$
			$\Theta$
			$\Pi$
			$\Omega$
		\section{数学函数}
			$\log \sin \cos \arcsin \arccos \ln$
			
			$\sqrt{x} \sqrt[4]{x^2 + 1}$
		\section{分式}
			$\frac{a+b}{a+c}$
		\section{行间公式}
			\subsection{美元符号}
				$$a+b=b+a$$
			\subsection{中括号}
				\[a+b=b+a\]
			\subsection{displaymath环境}
				\begin{displaymath}
					a+b=b+a
				\end{displaymath}
			\subsection{自动编号公式equation环境}
				交换律见式 \ref{eq:commutative} % 引用下面标签的内容
				\begin{equation}
					a+b=b+a \label{eq:commutative} % 把这行式子的编号保存到标签中
				\end{equation}
			\subsection{不编号公式equation*环境}
				交换律见式
				\begin{equation*}
					a+b=b+a
				\end{equation*}
				
				再如公式\ref{eq:pol}
				\begin{equation}
				x^5 - 7x^3 +4x = 0 \label{eq:pol}
				\end{equation}
\end{document}	% 正文结束
