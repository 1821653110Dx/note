\documentclass[a4paper, oneside]{ctexart}

\usepackage{geometry}
\geometry{
left=2.5cm.
right=2.5cm,
top=2.5cm,
bottom=2.5cm,
}

\usepackage{graphicx}
\graphicspath{{figures/},{pics/}}

\begin{document}
	\LaTeX{}中\TeX系统的吉祥物---小狮子见图\ref{fig-lion}	% \ref{fig-lion} = 引用fig-lion对应的对象的编号值
	\begin{figure}[htbp]	% 启用图像的浮动体环境(进行高级图像编辑需要开启~),浮动格式为htbp(一般情况下浮动格式就这么设置);为什么要设置浮动体环境:为了让排版更加紧密
		\centering	% 居中对齐
		\caption{\TeX 系统的吉祥物---小狮子}\label{fig-lion}	% 设置图片标题并自动编号,这个对象的标签为:fig-lion

		\includegraphics[scale=0.3]{lion}
	\end{figure}
	
	在\LaTeX{}中的表格如表\ref{tab-score}
	\begin{table}[htbp]	% 启用表格的浮动体环境,浮动格式为htbp
		\centering
		\caption{考试成绩}\label{tab-score}	% 设置表格标题并自动编号(与插图标题相互独立),这个对象的标签为:~
		
		\begin{tabular}{c|c|c|c|}
			\hline
			姓名 & 语文 & 数学 & 英语 \\
			\hline
			张三 & 80 & 100 & 89 \\
			\hline
			李四 & 90 & 98 & 20 \\
			\hline
			王五 & 78 & 90 & 77 \\
			\hline
		\end{tabular}
	\end{table}
\end{document}

