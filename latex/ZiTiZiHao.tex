\documentclass[12pt, a4paper, onneside]{ctexart}

\usepackage{geometry}
\geometry{
left=2.5cm,
right=2.5cm,
top=2.5cm,
bottom=2.5cm,
}

\begin{document}
% 英文字体族设置
	\textrm{Roman Family}	% 把{}中文本的字体族设置为罗马
	\textsf{Sans Serif Family}
	\texttt{Typewriter Family}
	
	\rmfamily Roman Family	% 把命令后面的文本的字体族设置为罗马
	\sffamily Sans Serif Family
	\ttfamily Typewriter Family
% 中文字体族设置(设置字体族为宋体相当于在word中把字体设置为宋体)	
	\songti 宋体
	\heiti 黑体
	\fangsong 仿宋
	\kaishu 楷书
	
% 英文字体粗细设置
	\textmd{Medium Series}	% 把{}中文本的粗细设置为中等
	\textbf{boldface Series} % 把{}中文本的粗细设置为粗
	
	\mdseries Medium Series
	\bfseries boldface series
% 中文字体粗细设置
	\textbf{粗体}
% 英文字体形状
	\textup{Upright Shape}	% 设置{}中文本形状’直立‘
	\textit{Italic Shape}	% 设置{}中文本形状为‘意大利斜体’
	\textsl{Slanted Shape}	% 设置{}中文本形状为’伪斜体‘
	\textsc{Small Caps Shape}	% 设置{}中文本形状为小型大写
% 中文字体形状设置
	\textit{斜体}
	\upshape Upright Shape
	\itshape Italic Shape
	\slshape Slanted Shape
	\scshape Small Caps Shape
% 英文字体大小设置
	{\tiny hello}\\
	{\scriptsize hello}\\
	{\small hello}\\
	{\normalsize hello}\\
	{\large hello}\\
	{\Large hello}\\
	{\LARGE hello}\\
	{\huge hello}\\
	{\Huge Hello}\\
% 中文字体大小设置
	\zihao{5} 你好!% 设置后面的中文字体大小为5号
\end{document}
